% --------------------------------------------------------------
% This is all preamble stuff that you don't have to worry about.
% Head down to where it says "Start here"
% --------------------------------------------------------------
 
\documentclass[12pt]{article}
 
\usepackage[margin=1in]{geometry} 
\usepackage{amsmath,amsthm,amssymb}
\usepackage[spanish]{babel} 
\usepackage[utf8]{inputenc}
\usepackage{txfonts}
\usepackage{dsfont}


 
\newcommand{\N}{\mathbb{N}}
\newcommand{\Z}{\mathbb{Z}}
 
\newenvironment{theorem}[2][Theorem]{\begin{trivlist}
\item[\hskip \labelsep {\bfseries #1}\hskip \labelsep {\bfseries #2.}]}{\end{trivlist}}
\newenvironment{lemma}[2][Lemma]{\begin{trivlist}
\item[\hskip \labelsep {\bfseries #1}\hskip \labelsep {\bfseries #2.}]}{\end{trivlist}}
\newenvironment{exercise}[2][Exercise]{\begin{trivlist}
\item[\hskip \labelsep {\bfseries #1}\hskip \labelsep {\bfseries #2.}]}{\end{trivlist}}
\newenvironment{reflection}[2][Reflection]{\begin{trivlist}
\item[\hskip \labelsep {\bfseries #1}\hskip \labelsep {\bfseries #2.}]}{\end{trivlist}}
\newenvironment{proposition}[2][Proposition]{\begin{trivlist}
\item[\hskip \labelsep {\bfseries #1}\hskip \labelsep {\bfseries #2.}]}{\end{trivlist}}
\newenvironment{corollary}[2][Corollary]{\begin{trivlist}
\item[\hskip \labelsep {\bfseries #1}\hskip \labelsep {\bfseries #2.}]}{\end{trivlist}}
 
\begin{document}
 
% --------------------------------------------------------------
%                         Start here
% --------------------------------------------------------------
 
%\renewcommand{\qedsymbol}{\filledbox}
 
%\title{Matrices}%replace X with the appropriate number
 
%\maketitle

\section*{Matrices}
 
 
Sea un sistema lineal de la forma.

\begin{equation}
  \label{eq:t}
  \begin{aligned}
A_{11}x_{1}+A_{12}x_{2}+ \cdots + A_{1n}x_{n} = y_{1} \\
A_{21}x_{1}+A_{22}x_{2}+ \cdots + A_{2n}x_{n} = y_{2} \\
 \vdots \quad + \quad \vdots  \quad + \quad + \vdots = \vdots \\
A_{m1}x_{1}+A_{m2}x_{2}+ \cdots + A_{mn}x_{n} = y_{m} 
  \end{aligned}
\end{equation}

Este se lo puede expresar en forma matricial. 

\begin{equation}
\varmathbb{A X} = \varmathbb{Y}
\end{equation}

Donde, 


\[
\varmathbb{A} = \begin{bmatrix}
	A_{11} & \cdot &  A_{1n}\\
    \vdots &  &  \vdots \\
   A_{m1} & \cdot & A_{mn} \\
\end{bmatrix}
\]

Es la matriz de coe$f$icientes del sistema. Donde cada $A_{ij}$, $i=1,...,m$ y $j=1,...n$. Creando asi una matriz \textit{m x n} sobre algun cuerpo K.
 
\[
\varmathbb{X} = \begin{bmatrix}
	X_{1} \\
    \vdots \\
   X_{n}  \\
\end{bmatrix}
\]

Es la matriz de n x 1 de incognitas. Finalmente,

\[
\varmathbb{Y} = \begin{bmatrix}
	Y_{1} \\
    \vdots \\
   Y_{m}  \\
\end{bmatrix}
\]

\section{Operaciones Elementales}

Para poder resolver este sistema lineal en forma matricial es importante considerar las siguientes operaciones elementales fila en una matriz n x m, sobre el cuerpo K. 

\begin{itemize}
\item Multiplicacion de una fila de A por un escalar no nulo c.
\item Remplazo de la r-\'esima fila de A por la fila r m\'as c veces la fila s, donde c es cualquier escalar no nulo y r $\neq$ s
\item Intercambio de dos filas de A. 
\end{itemize}

 
\end{document}